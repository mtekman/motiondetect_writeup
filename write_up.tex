% !TEX TS-program = pdflatex
% !TEX encoding = UTF-8 Unicode
\documentclass[11pt]{article} % use larger type; default would be 10pt
\usepackage[utf8]{inputenc} % set input encoding (not needed with XeLaTeX)

%%% PAGE DIMENSIONS
\usepackage{geometry} % to change the page dimensions
\geometry{a4paper} % or letterpaper (US) or a5paper or....

%%% PACKAGES
\usepackage{qtree} % Tree support
\usepackage{graphicx} % support the \includegraphics command and options
\usepackage{wrapfig} % Figure wrapping
% \usepackage[parfill]{parskip} % Activate to begin paragraphs with an empty line rather than an indent
\usepackage{booktabs} % for much better looking tables
\usepackage{array} % for better arrays (eg matrices) in maths
\usepackage{paralist} % very flexible & customisable lists (eg. enumerate/itemize, etc.)
\usepackage{verbatim} % adds environment for commenting out blocks of text & for better verbatim
\usepackage{subfig} % make it possible to include more than one captioned figure/table in a single float
\usepackage{url}
\usepackage{enumerate}
\usepackage{cleveref}  %cites figures intelligently
\usepackage{import} % document structuring
\usepackage{float}  %These two ensure that table position follows text by specifcing {table}[H]
\restylefloat{table}

%CODE LISTINGS
\usepackage{color}
\usepackage{listings}

\lstset{
	tabsize=4,
%	language=matlab,
        	basicstyle=\scriptsize,
%     	upquote=true,
       	aboveskip={\baselineskip},
        	columns=fixed,
        	showstringspaces=false,
        	extendedchars=true,
        	breaklines=true,
	prebreak = \raisebox{0ex}[0ex][0ex]{\ensuremath{\hookleftarrow}},
	frame=single,
        	showtabs=false,
        	showspaces=false,
        	showstringspaces=false,
        	identifierstyle=\ttfamily,
        	keywordstyle=\color[rgb]{0,0,1},
        	commentstyle=\color[rgb]{0.133,0.545,0.133},
        	stringstyle=\color[rgb]{0.627,0.126,0.941},
	language=C++
}

%%% HEADERS & FOOTERS
\usepackage{fancyhdr} % This should be set AFTER setting up the page geometry
\pagestyle{fancy} % options: empty , plain , fancy
\renewcommand{\headrulewidth}{0pt} % customise the layout...
\lhead{}\chead{}\rhead{}
\lfoot{}\cfoot{\thepage}\rfoot{}

%%% SECTION TITLE APPEARANCE
\usepackage{sectsty}
\allsectionsfont{\sffamily\mdseries\upshape} % (See the fntguide.pdf for font help)
%\ttfamily
%\sffamily\mdseries\upshape
\rmfamily
\usepackage{titlesec}
%\titleformat{\subsection}[runin]{\mdseries\bf}{\thesubsection}{1em}{}
%\titleformat{\subsubsection}[runin]{\mdseries\bf\underline\large}{\thesubsection}{1 em}{\vspace{-5 pt}}

\usepackage{footbib}

% (This matches ConTeXt defaults)

%%% ToC (table of contents) APPEARANCE
\usepackage[nottoc,notlof,notlot]{tocbibind} % Put the bibliography in the ToC
\usepackage[titles,subfigure]{tocloft} % Alter the style of the Table of Contents
\renewcommand{\cftsecfont}{\rmfamily\mdseries\upshape}
\renewcommand{\cftsecpagefont}{\rmfamily\mdseries\upshape} % No bold!
\newcommand{\tab}{\hspace*{2em}}


%%% END Article customizations

%%% The "real" document content comes below...

\title{\Huge Motion Detection and\\Other Imaging Operations for\\Debian-Based Mobile Devices }
\date{30 August 2012}
\author{{\bf By Mehmet Tekman}\\\small MSc Computer Science\\\small University College London\\\\
\large Supervisors: Robin Hirsch and Fabrizio Pece}


\begin{document}
\maketitle 

\part*{}{\tiny.\\\\\\\\\\\\\\\\}
\begin{abstract}
Morphological image processing techniques were adopted upon frames captured by the camera on the Nokia N900 smartphone to correctly detect motion for a wide range image sizes using the FCam API and CImg processing library in C++ within the Qt Framework.\\\tab A commandline interface was then developed to help facilitate time lapse operations in order to efficiently watch a target within a set period of time at specified intervals. The motion detection and time lapse components were merged into a single application dubbed ‘WatchDog’ and released successfully on public repositories.\\\tab IP webcam operations were also developed but they will be implemented into WatchDog at a further date.
\\\\\let\thefootnote\relax\footnote{This report is submitted as part requirement for the MSc Computer Science degree at UCL. It is substantially the result of my own work except where explicitly indicated in the text. The report may be freely copied and distributed provided the source is explicitly acknowledged.}
\end{abstract}
\pagebreak
\tableofcontents
\begin{center}
\vspace*{\fill}
{\bf Acknowledgements}\\
\end{center}
I would like to thank the following people:
	\begin{description}
	\item At UCL:
		\begin{description}
		\item [Robin Hirsch] For being stunningly flexible with my project idea, and for being incredibly helpful in finding me a secondary supervisor.
		\item [Fabrizio Pece] For the great discussions and the very detailed guidance into all aspects of the project, notably the motion detection parts.
		\end{description}
	\item At Home:
		\begin{description}
		\item [Family] For being notably patient and forgiving with my odd schedules and moods.
		\item [‘deed’] For being the first user on the Maemo forum to test my application upon release, and the glowing review he gave it.
		\end{description}		
	\end{description}
\let\thefootnote\relax\footnote{\underline{Note:} This write up contains colors that are best viewed from the PDF file on the CD}
\pagebreak

%\import{parts/}{background.tex}
%\import{parts/}{motiondetect.tex}
%\import{parts/}{timelapse.tex}
%\import{parts/}{ipstreamer.tex}
%\import{parts/}{userinterfaces.tex}
%\import{parts/}{release.tex}

\part{Bibliography}

\import{parts/}{appendix.tex}

\end{document}

Transcript of commits

Section: Future features
Plan to have a 'night mode' that can be used for detecting movement at night using a longer exposure on the CMOS chip